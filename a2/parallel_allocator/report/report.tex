\documentclass{article}
\usepackage{fullpage}
\usepackage{graphicx}
\usepackage{listings}
\usepackage{framed}
\usepackage{multicol}
\usepackage{hyperref}
\usepackage{float}

\author{Andrei Soltan\\998556067\\g0soltan@cdf.toronto.edu
\and Jonathan Prindiville\\993177628\\g2prindi@cdf.toronto.edu}
\title{CSC469: Assignment 2}
\date{15 March, 2013}

% Show paragraphs in table of contents
\setcounter{tocdepth}{5}

\lstset{
    mathescape=false,
    basicstyle=\small,
    stringstyle=\ttfamily, % typewriter type for strings
    showstringspaces=false } % no special string spaces

\begin{document}

\maketitle

\tableofcontents

\newpage
\section{Introduction}

The goal of this assignment is to implement a multithreaded memory allocator.
The challenge is to achieve an implementation that is as fast as a regular 
malloc, and whose performance scales linearly with the increased number of
threads, while at the same time avoiding false sharing and having low 
memory fragmentation. For the approach and the design of the allocator, 
We use as point of reference the Hoard allocator paper \cite{berger00}.
In addition, we have used the description and discussion of the Miser allocator
by Intel \cite{miser-intel}.

We proceed to describe the design of our allocator in section \ref{sec:design}. 
We discuss the decision points in our design and some of the alternatives we 
consided during both implementation and design in section \ref{sec:alternatives}.
Section \ref{sec:performance} discusses the performance details of the allocator
and shows the comparison with other allocators. Finally, section 
\ref{sec:conclusion} presents our conclusions on this multithreaded allocator.

\newpage
\section{Allocator Design}
\label{sec:design}

Insert smart text about the design of our allocator...

What metadata do we maintain? Why do we need it?
\begin{itemize}
	\item for the allocator
	\item for the heap(s)
	\item for the superblock(s)
	\item for the blocks inside the superblock
\end{itemize}

Where do we put the metadata? Why? (e.g. the structs for ht efree list at the 
end of the free memory block, because that makes it easier to merge to the 
right, where there is usally free memory. Just an example, not necessarily true.)

How do we handle rhe really large allocations?

Was there something we did to boost performance? (mention memory usage, 
processing, internal and external fragmentation (with no numbers?))

What other questions do we want to answer or design decisions that we think
are worth mentioning?

\newpage
\section{Design Alternatives}
\label{sec:alternatives}

Insert smart text about other features we considered for our allocator...

Was there something that would work better but is too hard to develop?

Was there something that was easier to develop but would give slightly slower
performance?

Was there something we tried, but found that it wouldn't work well (or took too 
much time/effort), and decided in favor of a simpler solution?

Was there something we did, avoided or fixed to have good performance?

Did we decided in favor or against something regarding fragmentation?

Should this a subsection of the Design section above?

\newpage
\section{Performance Analysis}
\label{sec:performance}

Insert smart text about how well our allocator performs...

\textbf{MUST HAVE} or we lose marks for not mentioning (from grading rubric)
\begin{itemize}
	\item Sequential Speed
	\item Scalability
	\item False Sharing Avoidance
	\item Fragmentation
\end{itemize}

Describe the experimental setup. (Subsection?)

Describe the performance measurements. \textbf{Must} have plenty of numbers 
(i.e. not "very few" according to rubric). (Subsection?)

Describe the memory usage, utilization, overhead, internal and external 
fragmentation. Pay significant attention to \textbf{fragmentation} and 
\textbf{usage}. Look at Hoard paper, performance section. 

\newpage
\section{Conclusion}
\label{sec:conclusion}

Insert smart text about how we did awesome... This might take work. :)

\newpage

% Declare the bibligraphy to use one-digit indexes.
\begin{thebibliography}{9}
	
	\bibitem{berger00}
		Berger, Emery D., et al.
		"Hoard: A scalable memory allocator for multithreaded applications."
		\textit{ACM SIGPLAN Notices} 35.11 
		(2000): 
		117-128.
	\bibitem{miser-intel}
		Lewin, Stephen.
		"Miser - A Dynamically Loadable Memory Allocator for Multi-Threaded Applications."
		\textit{Intel Developer Zone.}
		Intel, 
		n.d. Web. 10 Mar. 2013.

\end{thebibliography}

\end{document}
