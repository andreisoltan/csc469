\documentclass[12pt]{article}
\usepackage{fullpage}
%\usepackage{graphicx}
\usepackage{listings}
\usepackage{framed}
%\usepackage{multicol}
\usepackage{hyperref}
\usepackage{float}
\usepackage{enumitem}

\newcommand{\mono}[1]{\texttt{#1}}

\lstset{ %
    basicstyle=\footnotesize\ttfamily,
    frame=single,
    numbers=none,
    language=C,
    xleftmargin=0.5in,
    xrightmargin=0.5in,
    captionpos=b,
    breaklines=true,
    breakatwhitespace=true
}

\setdescription{ %
    leftmargin=\parindent,
    labelindent=\parindent,
    itemsep=0pt,
    topsep=0pt
}

\author {Jonathan Prindiville\\993177628\\g2prindi@cdf.toronto.edu
\and Andrei Soltan\\998556067\\g0soltan@cdf.toronto.edu}
\title{CSC469: Assignment 3}
\date{5 April, 2013}

% Show paragraphs in table of contents
\setcounter{tocdepth}{5}

%\lstset{
%    mathescape=false,
%    basicstyle=\small,
%    stringstyle=\ttfamily, % typewriter type for strings
%    showstringspaces=false } % no special string spaces
%
\begin{document}

\maketitle

\tableofcontents

\newpage
\section{Introduction}
This document discusses the design and implementation of our
chat client. Our design is simple -- heavily influenced by the
starter code -- and effective -- handling all of the faults discussed
in the handout. In addition, we make an effort to deal with transient
network errors by attempting to resend the offending communication.

\begin{itemize}
\item Section \ref{sec:basic} gives an overview of the execution of the client.
\item Section \ref{sec:init} elaborates on the connection process and
re-establishment of a lost connection.
\item Section \ref{sec:keepalive} details our strategy for maintaining
the connection with the server.
\item Section \ref{sec:util} explains some functions and macros which
are important to our implementation.
\item Section \ref{sec:build} is a listing of build parameters affecting
the operation of the client.
\end{itemize}

\section{Basic operation}
\label{sec:basic}
We've kept the interface as well as the basic structure
 of the client's operation from the starter code which was
provided to us. The user must supply a name on the command-line and 
also, if the \mono{USE\_LOCN\_SERVER}
flag was not present at build time, the server address and port numbers.

Once the connection parameters have been collected, we start the
receiver process. The existing xterm-launching infrastructure has been
maintained. Once the receiver is up and running it begins listening
on a UDP port of the operating system's choosing and reports the port
number to the client process through their shared message queue.

After the receiver process has been set up, we try to connect to the
server and then enter our main loop. A non-zero return value from the
main loop causes us to attempt reconnection to the server. Repeated 
failure of the connection attempts will cause us to exit.

Once connected and registered we can send UDP chat messages from the
client to the server, and the receiver process will listen for incoming
chat messages from the server to display in its xterm.

Our main loop is quite simple, much of it maintained from the starter code
as mentioned earlier. We check the time of last contact with the server,
deciding whether or not to send a keepalive message. We then
poll standard input to see if the user has entered anything, copying it into
a buffer if available, until we've accumulated a whole line. The line will be
dispatched to \mono{handle\_chatmsg\_input} or
\mono{handle\_command\_input}. If there 
was nothing to read from standard input, we check for messages in the
queue from the receiver process. All of the calls here are non-blocking --
the only time our client chooses to block is when sleeping between retry
attempts.

\section{Connection establishment}
\label{sec:init}
The function \mono{init\_client} does the work of connecting to and registering
on the chat server for us.
If built with \mono{USE\_LOCN\_SERVER} our first order of business is to contact
the location server using \mono{retrieve\_chatserver\_info}. As with 
\mono{retry\_handler} (see sec \ref{func:retry}), \mono{retrieve\_chatserver\_info} can
make multiple attempts
at its goal before returning, retrying if it encounters errors in
\mono{case\_RETRYABLE} (see sec \ref{def:case}) and returning if it either runs
out of retry attempts or if it encounter an error that is not in
\mono{case\_RETRYABLE}.

We wrap our call to \mono{handle\_register\_req} with \mono{retry\_handler} and
also catch the errors \mono{NAME\_IN\_USE} and \mono{SERVER\_FULL}. In the event
of a different
error we try to contact the location server for new connection parameters.
\mono{NAME\_IN\_USE} is handled by prepending an underscore to the current username
before re-attempting the connection, and \mono{SERVER\_FULL} causes us to quit.

\subsection{Connection re-establishment}
In the event that we become disconnected and successfully re-establish a
connection with the server, our client will check to see if we were in a chat
room before being disconnected. If so, we attempt to put the user back into the
same room using \mono{rejoin(char)}.

We first try simply changing to
the room -- if we get back \mono{ZERO\_ROOMS} or \mono{ROOM\_NOT\_FOUND} we will
try to creating the room. If the response to the creation request is
\mono{COMMAND\_SUCC} or \mono{ROOM\_EXIST} we try again to switch to that room.
In the case of any other returns (\mono{MAX\_ROOMS},
\mono{ROOM\_NAME\_TOOOO\_LOOOONG}, or another) we simply give up.

\section{Connection maintenance}
\label{sec:keepalive}
We wish to maintain the connection with the server even if our user is not
actively issuing commands. In order to do this, we arrange for 
\mono{MEMBER\_KEEP\_ALIVE} messages to be sent during times of inactivity.

As mentioned above, the main input loop will check the current time
against the recorded time of the last contact with the server. If it has
been more than \mono{KA\_MINUTES} minutes, we send a keepalive message.
This message
is sent using \mono{retry\_handler} (see sec \ref{func:retry}) so it will make
several attempts if there seem to be transient network issues.

In order to minimize the number of keepalive messages that we send, we've
also modified the receiver process to notify the client every time that it
receives a chat message from the server. The receiver opens the message queue
read/write rather than read-only as shown in the starter code.
Upon notification from the receiver, the client updates its record of last
contact with the server.

In this way, clients connected to an active chat will not need to send 
keepalive messages while those in inactive chat sessions will still have their
connection to the server tested at regular intervals.

\section{Utilities}
\label{sec:util}

\subsection{Macros}
See \mono{client.h}.

\subsubsection{\mono{case\_RETRYABLE}}
\label{def:case}
This is a series of case labels for use inside a switch statement intended
to represent errors which may be transient -- that is errors in response to
which we will simply try to send the message again. We use this in
\mono{retry\_handler} and \mono{retrieve\_chatserver\_info}.

As of this writing, \mono{case\_RETRYABLE} contains the following error values:
\mono{EAI\_AGAIN, EAGAIN, EPROTO, EHOSTUNREACH, ENETDOWN, ECOMM, ENETRESET,
ENETUNREACH}.

\subsubsection{\mono{case\_RETURN}}
Similar to \mono{case\_RETRYABLE} (sec \ref{def:case}), this is a series of case
labels for use in a switch statement. These represent errors which will not
be overcome be resending the message. These are immediately returned up the
call chain and will prompt an attempt to reconnect to the server. We use
this in \mono{retry\_handler} and \mono{handle\_command\_input}.

As of this writing, \mono{case\_RETURN} contains the following error values:
\mono{ECONNREFUSED, ECONNABORTED, ECONNRESET, ENOTCONN, ID\_INVALID}.

\subsection{Functions}

\subsubsection[\mono{retry\_handler}]{\mono{retry\_handler(int (*handler)(char*), char *arg, int *retries, int *pause)}}
\label{func:retry}
This function is used to wrap all of the functions dealing with command
messages. The motivation is that we would like to have a simple way to 
retry communication attempts that encounter transient errors. The errors
which trigger a retry can be seen in \mono{case\_RETRYABLE}
(sec \ref{def:case}). Any errors not in that list are passed immediately
up to the caller.

Successive retries will double the \mono{*pause} value.

Because \mono{*retries} and \mono{*pause} are passed as pointers, and
modified in-place, the caller can see
how many attempts were taken and has the option of handling special error
cases before passing back in to \mono{retry\_handler}.

A typical usage of \mono{retry\_handler} is shown below, taken from
\mono{rejoin()}:
\begin{lstlisting}[caption=Excerpt from rejoin()]
    /* Try switching to the room we were in */
    ret = retry_handler(handle_switch_room_req,
        server_room_name, &retries, &pause);

    switch (ret) {
        case COMMAND_SUCC:
            /* cool, done */
            return ret;
        case ZERO_ROOMS:
        case ROOM_NOT_FOUND:
            /* ~~~ omitted room creation code ~~~ */
            break;
        case ROOM_FULL:
            /* can't join */
            ret = 0;
            break;
        default:
            /* some other problem */
            fprintf(stderr, "Could not rejoin room '%s'.\n",
                server_room_name);
            break;
    }

    return ret;
\end{lstlisting}

We wish to invoke \mono{handle\_switch\_room\_req}.
\mono{retry\_handler} will perform the call for us and if it encounters errors
covered by \mono{case\_RETRYABLE} it will retry. Any other values
returned by \mono{handle\_switch\_room\_req} will be returned up to
the caller. In this example, we see that \mono{rejoin()} handles a
few other return values from \mono{handle\_switch\_room\_req} and passes
defaults to passing other return values up to its caller.


\subsubsection[\mono{generic\_command}]{\mono{generic\_command(int msg\_type, char *arg, char *buf)}}
Implements the general case of command message sending and receiving. All of
the command message handlers are implemented using this function. 

Return values are as follows:
\begin{description}
    \item[some errno value] if any of our calls fail and set errno, we pass it back to our caller
    \item[\mono{INVALID\_ID}] if the server does not recognize our member id. This
    will trigger a reconnection attempt.
    \item[\mono{COMMAND\_SUCC}] on success
    \item[\mono{COMMAND\_FAIL}] on failure
    \item[\mono{BOGUS\_RESPONSE}] on an unexpected reply
\end{description}

A typical usage of \mono{generic\_command}, the implementation of
\mono{handle\_room\_list\_req}, appears below:
\begin{lstlisting}[caption=handle\_room\_list\_req()]
    int ret = generic_command(ROOM_LIST_REQUEST, NULL, buf);
    switch(ret) {
        case COMMAND_SUCC:
            printf("%s\n", (char *) (cmh->msgdata));
            break;
        case COMMAND_FAIL:
            printf("Could not list rooms: %s\n", (char *) (cmh->msgdata));
            break;
        default:
            return ret;
    }
    return 0;
\end{lstlisting}

This is a simple usage -- simply displaying a response for the user
-- but other handlers are more involved. When
\mono{handle\_create\_room\_req}, for instance, encounters a return value
of \mono{COMMAND\_FAIL} it will examine the message body and ultimately
return to its caller one of the more specific errors \mono{MAX\_ROOMS},
\mono{ROOM\_EXISTS} or \mono{ROOM\_NAME\_TOOOO\_LOOOONG}.


\section{Build parameters}
\label{sec:build}
\subsection{\mono{RETRY\_COUNT}}
Number of retry attempts. Default value is 4.
\subsection{\mono{RETRY\_PAUSE}}
Pause (in seconds) between retry attemts, usually this value is backed off
exponentially for subsequent retries. Default value is 1.
\subsection{\mono{KA\_MINUTES}}
Time (in minutes) without contact with the server before we will send a
\mono{MEMBER\_KEEP\_ALIVE} message. Default value is 1.
\end{document}
