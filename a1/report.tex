\documentclass{article}
\usepackage{fullpage}
\usepackage{graphicx}
\usepackage{listings}
\usepackage{framed}
\usepackage{multicol}
\usepackage{hyperref}

\author{Andrei Soltan \and Jonathan Prindiville}
\title{CSC469: Assignment 1}
\date{11 February, 2013}

% Show paragraphs in table of contents
\setcounter{tocdepth}{5}

\lstset{
    basicstyle=\small,
    stringstyle=\ttfamily, % typewriter type for strings
    showstringspaces=false } % no special string spaces

\begin{document}

\maketitle

\tableofcontents

\newpage
\section{Introduction}
Our goal in this assignment is to develop our understanding of performance
measurement and benchmarking as it relates to operating systems -- Linux 3.2
in particular.

Section~\ref{sec:A} contains our exploration of timer interrupts and context
switching between two processes.

Section~\ref{sec:B} will attempt to examine the effect of measurement bias
on several programs, bzip2, lbm and perlbench.

\section{TODO}
- platform data for b3195 workstatins
- histogram data / graph / explanation choosing threshold
- inactive periods data / graph / explanation of choice
- questions

\newpage
\section{Part A: Performance measurement}
\label{sec:A}

(With what frequency do timer interrupts occur? How long does it take to handle a timer interrupt?)
(If it appears that there are other, non-timer interrupts (that is, other short periods of inactivity that don't fit the pattern of the periodic timer interrupt), explain what these are likely to be, based on what you can determine about other activity on the system you are measuring.)
(Over the period of time that the process is running (that is, without lengthy interruptions corresponding to a switch to another process), what percentage of time is lost to servicing interrupts (of any kind)?)

\subsection{Tracking process activity}

\subsubsection{Methodology}
In order to answer the questions that we have, we will be using data collected
from the Pentium TSC register. We'll use the program \lstinline{parta} in order
to collect that data. We first must establish a threshold for the timer
measurements -- above which we will assume that the process has been inactive.
After finding that threshold value, we'll be able to use \lstinline{parta} to
produce a plot showing a process' active and inactive periods over time.

To reduce the number of potentially conflicting processes running on the same
machine, we've written a script (see~\ref{tool:find-empty}) to find empty
workstations in the CDF laboratories. Because we hope to extract measurements
in milliseconds, we are choosing to avoid those workstations which use cpufreq
to dynamically set their clock speed.

\subsubsection{Results}
(Insert tabular, graphical results, discussion)

\paragraph{Threshold}
In order to find our inactivity threshold we will be using
\lstinline{parta -i -n <num>}. In this mode of operation, we collect
\lstinline{<num>} timer samples in a tight loop and output the difference
between successive samples. Sorting this output and counting intervals of the
same length \footnote{Using \lstinline{sort}, \lstinline{uniq -c}, and
\lstinline{awk}} we will get output like the following, where the left column
represents interval length and the right, number of intervals:
\begin{framed}
    \label{lst:intervals}
    \lstinputlisting[multicols=3]{intervals.log}
\end{framed}

It is clear from the above that the vast majority (\textgreater~99\%) of the
intervals are on the order of one hundred cycles in length. Depending on the
machine, the next smallest interval length tends to be in the 3000-6000 cycle
range. Accordingly, we will use a threshold of 600 cycles in our next
experiment to detect process inactivity.

\paragraph{Process inactivity}
In order to chart process activity

\begin{center}
    \begin{figure}[h]
        \caption{Inactive (grey) and active (red) periods of execution}
        \input{inactive_periods-ms}
    \end{figure}
\end{center}
(include hardware platform, cpu family, speed, cache, memory, disk params, anything relevant)


\subsection{Context switches}
(How long is a time slice? That is, how long does one process get to run before it is forced to switch to another process?)
(Is the length of the time slice affected by the number of processes that you are using?)
(Are you surprised by your measurements?)
(How does it compare to what you were told about time slices in your previous OS course?)
\subsubsection{Methodology}
(include hardware platform, cpu family, speed, cache, memory, disk params, anything relevant)
\subsubsection{Results}
(Insert tabular, graphical results, discussion)
\begin{figure}[h]
    \caption{Inactive (grey) and active periods of execution for parent (red)
    and child (blue) processes}
    \input{context_switch-ms}
\end{figure}

\newpage
\section{Part B: Measurement bias}
\label{sec:B}
(How significant is measurement bias due to Unix environment size on current systems?)
(To what extent does Linux ASLR (address space layout randomization) affect measured performance? Does it mitigate or exacerbate the effect of Unix environment size?)
(What underlying architectural events in the microprocessor appear to be causing the measurement bias?)
\subsection{Methodology}
(include hardware platform, cpu family, speed, cache, memory, disk params, anything relevant)
\subsection{Results}
(Insert tabular, graphical results, discussion)

\newpage
\appendix
\section{Tools}
(Include an appendix that describes how to use your benchmark tools so that someone else could reproduce your experiments.)

\subsection{count-users.sh}
Used a proof of concept for find-empty-wkstn. Counts the number of currently
logged in users.

\subsection{find-empty-wkstn.sh}
\label{tool:find-empty}
Used to identify CDF workstations that are currently unnocupied. We used free
lab machines to run many of our tests. A little work with \lstinline{ping}
revealed the number of machines in each lab and this script simply loops over
the machines in a specified lab, checking them with \lstinline{count-users.sh}.

\subsection{platform-info.sh, gather-platform-info.sh}
\label{tool:platform}
Used to get a sense of the hardware available in each of the CDF labs. Of particular interest was the use of dynamic cpu frequency adjustments -- something that we were trying to avoid. 

\begin{tabular}{ | c | c | c | c | c | c |}
\hline
LAB   & MODEL NO      & Max Frq & N Cores & RAM  & cpufreq governor \\
\hline 
b2200 & C2Duo E6550   & 2.33GHz & 2 cores &  2GB & ondemand \\
b2220 & Pentium G630  & 2.70GHz & 2 cores &  8GB & ondemand \\
b2240 & Core i5 3570  & 3.40GHz & 4 cores & 20GB & ondemand \\
b3185 & Pentium E2160 & 1.80GHz & 2 cores &  2GB & ondemand \\
b3200 & Core i5 650   & 3.20GHz & 4 cores &  1GB & ondemand \\
\hline
b2210 & P4            & 3.20GHz & 2 cores &  1GB & none \\
b3175 & C2 6300       & 1.86GHz & 2 cores &  1GB & none \\
b3195 & C2 6300       & 1.86GHz & 2 cores &  1GB & none \\
s2360 & C2 6300       & 1.86GHz & 2 cores &  1GB & none \\
\hline
\end{tabular}

\subsection{plot-histogram.sh}
Used to determine a suitable threshold value for our other experiments.

\subsection{plot-parta.awk}
Transforms output from part a trials (inactive periods, context switching) into gnuplot input.


%Check out the random bars in figure \ref{fig:random}.
%\begin{figure}
%\centering
%\includegraphics[scale=1.25]{random.eps}
%\caption{Random intervals (red).}
%\label{fig:random}
%\end{figure}

\end{document}
